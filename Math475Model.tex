\documentclass[11pt]{article}\usepackage[]{graphicx}\usepackage[]{color}
% maxwidth is the original width if it is less than linewidth
% otherwise use linewidth (to make sure the graphics do not exceed the margin)
\makeatletter
\def\maxwidth{ %
  \ifdim\Gin@nat@width>\linewidth
    \linewidth
  \else
    \Gin@nat@width
  \fi
}
\makeatother

\definecolor{fgcolor}{rgb}{0.345, 0.345, 0.345}
\newcommand{\hlnum}[1]{\textcolor[rgb]{0.686,0.059,0.569}{#1}}%
\newcommand{\hlstr}[1]{\textcolor[rgb]{0.192,0.494,0.8}{#1}}%
\newcommand{\hlcom}[1]{\textcolor[rgb]{0.678,0.584,0.686}{\textit{#1}}}%
\newcommand{\hlopt}[1]{\textcolor[rgb]{0,0,0}{#1}}%
\newcommand{\hlstd}[1]{\textcolor[rgb]{0.345,0.345,0.345}{#1}}%
\newcommand{\hlkwa}[1]{\textcolor[rgb]{0.161,0.373,0.58}{\textbf{#1}}}%
\newcommand{\hlkwb}[1]{\textcolor[rgb]{0.69,0.353,0.396}{#1}}%
\newcommand{\hlkwc}[1]{\textcolor[rgb]{0.333,0.667,0.333}{#1}}%
\newcommand{\hlkwd}[1]{\textcolor[rgb]{0.737,0.353,0.396}{\textbf{#1}}}%
\let\hlipl\hlkwb

\usepackage{framed}
\makeatletter
\newenvironment{kframe}{%
 \def\at@end@of@kframe{}%
 \ifinner\ifhmode%
  \def\at@end@of@kframe{\end{minipage}}%
  \begin{minipage}{\columnwidth}%
 \fi\fi%
 \def\FrameCommand##1{\hskip\@totalleftmargin \hskip-\fboxsep
 \colorbox{shadecolor}{##1}\hskip-\fboxsep
     % There is no \\@totalrightmargin, so:
     \hskip-\linewidth \hskip-\@totalleftmargin \hskip\columnwidth}%
 \MakeFramed {\advance\hsize-\width
   \@totalleftmargin\z@ \linewidth\hsize
   \@setminipage}}%
 {\par\unskip\endMakeFramed%
 \at@end@of@kframe}
\makeatother

\definecolor{shadecolor}{rgb}{.97, .97, .97}
\definecolor{messagecolor}{rgb}{0, 0, 0}
\definecolor{warningcolor}{rgb}{1, 0, 1}
\definecolor{errorcolor}{rgb}{1, 0, 0}
\newenvironment{knitrout}{}{} % an empty environment to be redefined in TeX

\usepackage{alltt}

% These packages help with equations, math symbols, etc.
\usepackage{amsmath,amssymb}
% This EXCELLENT resource is your guide to writing equations in LaTeX!
%  ftp://ftp.ams.org/pub/tex/doc/amsmath/short-math-guide.pdf

\usepackage[margin=1in]{geometry} % better margins and spacing
\setlength{\parindent}{0in} % 0 = no paragraph indentation 

% For more compact numbered lists...
\usepackage{enumitem}
\setlist{noitemsep, topsep=-0.5em}

\usepackage{multicol} % Easy columns

\usepackage{graphicx} % If you want to include an image
\IfFileExists{upquote.sty}{\usepackage{upquote}}{}
\begin{document}
% Configure some details for knitr...


% Main text begins here:
\textbf{\large COURSE TITLE, Homework X \hfill STUDENT NAME}  % Replace Solutions with your name.
\vspace{1pt}
\hrule
~\\ % this just forces a blank line. 

\textbf{1.} What is $1+1$? 

$$1+1=2$$

Or in R... %% This is a block of R code parsed by knitr!
\begin{knitrout}\footnotesize
\definecolor{shadecolor}{rgb}{0.969, 0.969, 0.969}\color{fgcolor}\begin{kframe}
\begin{alltt}
\hlcom{## R code to compute 1+1 (the complicated way...)}
\hlstd{x} \hlkwb{=} \hlnum{1}
\hlstd{y} \hlkwb{=} \hlnum{1}
\hlstd{addtogether} \hlkwb{=} \hlkwa{function}\hlstd{(}\hlkwc{a}\hlstd{,} \hlkwc{b}\hlstd{) \{}
    \hlkwd{return}\hlstd{(a} \hlopt{+} \hlstd{b)}
\hlstd{\}}
\hlkwd{addtogether}\hlstd{(x, y)}
\end{alltt}
\begin{verbatim}
## [1] 2
\end{verbatim}
\end{kframe}
\end{knitrout}
~\\

\textbf{2.} Use standard, and robust regression methods to fit a linear model to these data:

\begin{table}[h!]  \begin{center}
 \begin{tabular}{c|cccccccccccccc}
\textbf{x} & 17 & 6 & 5 & 3 & 9 & 10 & 20 & 14 & 1 & 2 & 11 & 15 & 8 & 4 \\ \hline
\textbf{y} & 13.1& 6.3& 2.74& 5.13& 4.99& 7.86& 19.20& 18.3& 1.59& 2.11& 9.83& 16.3& 2.2& 5.02 \\
 \end{tabular}  \end{center}
\end{table}

\begin{knitrout}\footnotesize
\definecolor{shadecolor}{rgb}{0.969, 0.969, 0.969}\color{fgcolor}\begin{kframe}
\begin{alltt}
\hlcom{## R code to do the regression}
\hlkwd{library}\hlstd{(robustbase)} \hlcom{# load for lmrob(). To install run install.packages("robustbase") in R}
\hlcom{# See echo=... above, which hides this and the next two lines}
\hlstd{fitrob}\hlkwb{=}\hlkwd{lmrob}\hlstd{(y}\hlopt{~}\hlstd{x)}
\hlkwd{par}\hlstd{(}\hlkwc{mar}\hlstd{=}\hlkwd{c}\hlstd{(}\hlnum{4}\hlstd{,}\hlnum{4}\hlstd{,}\hlnum{2}\hlstd{,}\hlnum{1}\hlstd{),} \hlkwc{oma}\hlstd{=}\hlkwd{c}\hlstd{(}\hlnum{0}\hlstd{,}\hlnum{0}\hlstd{,}\hlnum{0}\hlstd{,}\hlnum{0}\hlstd{))} \hlcom{# see echo=... above, which hides this line.}
\hlkwd{plot}\hlstd{(x,y,}\hlkwc{pch}\hlstd{=}\hlnum{19}\hlstd{)}
\hlkwd{abline}\hlstd{(fit,}\hlkwc{lty}\hlstd{=}\hlnum{1}\hlstd{)}
\hlkwd{abline}\hlstd{(fitrob,}\hlkwc{lty}\hlstd{=}\hlnum{2}\hlstd{)}
\hlkwd{legend}\hlstd{(}\hlstr{"topleft"}\hlstd{,}\hlkwd{c}\hlstd{(}\hlstr{"Simple Regression"}\hlstd{,}\hlstr{"Robust Regression"}\hlstd{),} \hlkwc{lty}\hlstd{=}\hlkwd{c}\hlstd{(}\hlnum{1}\hlstd{,}\hlnum{2}\hlstd{),} \hlkwc{cex}\hlstd{=}\hlnum{0.8}\hlstd{)}
\end{alltt}
\end{kframe}
\includegraphics[width=.45\linewidth]{figure/pigprofit-1} 
\includegraphics[width=.45\linewidth]{figure/pigprofit-2} 

\end{knitrout}

% For more examples, see the internet.
\begin{center}
\includegraphics[width=1 \textwidth]{p3p31}
\end{center}

\end{document}
